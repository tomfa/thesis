\section{Challenges} \label{sec:problem_challenges}

In this section, we will look at some of the challenges that each role face today when creating hybrid games, such as Don't Panic. These will strongly influence us when making high level requirements in the next section.

\subsection{Game Designer}
The work of the Game Designer is creating an entertaining experience. It's a creative process, with few rules that dictates how things ought to be. The challenges for a game designer is finding the inspiration and ideas necessary to create a good game concept. Even with a good concept, the designer should find volunteers to play the game and give adjustive feedback. Once a prototype is polished enough for the designer to be satisfied, the rules and concepts must be defined in such a way that the developer is able to translate it to a program.

\begin{itemize}
\item Finding inspiration for the game concept and ideas can be hard.
\item The game designer must find volunteers that can provide user feedback to help refine the game.
\item The game designer should preferably have knowledge of typical game concepts.
\item Defining the game clearly enough to be computer translated.
\end{itemize}

\subsection{Developer}
There exists few suitable digital tokens for hybrid board games. As with the augmented version of Don't Panic, this can lead to a large amount of time being spent finding or making custom tokens (number 2), and establishing communication (number 3 and 4) between the tokens. This also requires knowledge of low level programming. 

There are few existing game tools aimed for board games. Among them, there are no tools geared for using tangible tokens. A developer might therefore have to build the board game concepts from scratch (4, 5), and modify the tools to support the tangible aspect (4).

\begin{itemize}
\item Developers must know both high level and low level code
\item Large amount of time is used creating custom hardware tokens
\item Communication between tokens must be built from scratch.
\item Existing game tools is likely to require modification to support the tangible aspect of hybrid board games.
\item End-users (Players) use various devices, operating systems and screen sizes. It can be time consuming to support the different devices.
\item Licenses for game development tools can be costly.
\end{itemize}

\subsection{Player}
Challenges of the Player: First, the lack of mature components make it hard to initiate the game (number 4 in). In Don't Panic, set up of the game required technical knowledge and was time consuming. It required starting an off-site server, and starting scripts manually on some of the game tokens. Second, the equipment used was custom made, and could as such not be easily replaced (number 2).  Since the equpment was custom made, it could not be reused for other purposes without reprogramming. If a Player wish to acquire a second similar game, he must anticipate to buy another set of hardware (number 2). Having located and enjoyed one augmented board game, the lack of a an community or platform around hybrid board games can make it difficult to locate new such games (number 1)

\begin{itemize}
\item In existing prototypes it has been hard for players to initiate a game, as the setup have required techical knowledge.
\item Hardware purchased for one game has not been suitable for reuse in other games without reprogramming, which is both time consuming and requires technical knowledge.
\item Lack of community around hybrid board games, makes it hard to acquire and locate such games.
\end{itemize}